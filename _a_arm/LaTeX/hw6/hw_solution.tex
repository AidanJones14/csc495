
The linear state space equations can be derived in two different ways: (1) directly from the linearized equations of motion, and (2) by linearizing the nonlinear equations of motion.

Starting with the feedback linear state space equation in Equation~\eqref{eq:arm_linearized_eom_feedback_linarization} and solving for $\ddot{\theta}$ gives
\[
\ddot{\theta} = \frac{3}{m\ell^2}\tilde{\tau} - \frac{3b}{m\ell^2}\dot{\theta}.
\]
Therefore,
\begin{align*}
\dot{x} &\defeq \begin{pmatrix} \dot{x}_1 \\ \dot{x}_2 \end{pmatrix} 
= \begin{pmatrix}\dot{\theta} \\ \ddot{\theta}\end{pmatrix} 
= \begin{pmatrix} \dot{\theta} \\ \frac{3}{m\ell^2}\tilde{\tau} - \frac{3b}{m\ell^2}\dot{\theta}  \end{pmatrix}
=\begin{pmatrix} x_2 \\ \frac{3}{m\ell^2}\tilde{u} - \frac{3b}{m\ell^2}x_2  \end{pmatrix} \\
&= \begin{pmatrix} 0 & 1 \\ 0 & -\frac{3b}{2\ell^2} \end{pmatrix} \begin{pmatrix} x_1 \\ x_2 \end{pmatrix} + \begin{pmatrix} 0 \\ \frac{3}{m\ell^2} \end{pmatrix} \tilde{u}.
\end{align*}
Assuming that the measured output of the system is $y=\theta$, the linearized output is given by
\[
y = \theta = x_1 = \begin{pmatrix} 1 & 0\end{pmatrix}\begin{pmatrix}x_1 \\ x_2 \end{pmatrix} + (0)\tilde{u}.
\]
Therefore, the linearized state space equations are given by
\begin{align}
\dot{x} &= \begin{pmatrix} 0 & 1 \\ 0 & -\frac{3b}{2\ell^2} \end{pmatrix} x + \begin{pmatrix} 0 \\ \frac{3}{m\ell^2} \end{pmatrix} \tilde{u} 
\notag \\
y &= \begin{pmatrix} 1 & 0\end{pmatrix} x.
\label{eq:dm_arm_linearized_ss}
\end{align}

Alternatively, the state space equations can be found directly from the nonlinear equations of motion given in Equation~\eqref{eq:sm_arm_nonlinear_eom}, by forming the nonlinear state space model as
\begin{align*}
\dot{x} &\defeq \begin{pmatrix} \dot{x}_1 \\ \dot{x}_2 \end{pmatrix} 
= \begin{pmatrix}\dot{\theta} \\ \ddot{\theta}\end{pmatrix} 
= \begin{pmatrix} \dot{\theta} \\ \frac{3}{m\ell^2}\tau - \frac{3b}{m\ell^2}\dot{\theta} - \frac{3g}{2\ell}\cos\theta  \end{pmatrix} \defeq f(x,u).
\end{align*}
Evaluating the Jacobians at the equilibrium $(\theta_e, \dot{\theta}_e, \tau_e) = (0, 0, \frac{mg\ell}{2})$ gives
\begin{align*}
A &= \frac{\partial f}{\partial x}\Big|_e = \begin{pmatrix} \frac{\partial f_1}{\partial \theta} & \frac{\partial f_1}{\partial \dot{\theta}} \\ \frac{\partial f_2}{\partial \theta} & \frac{\partial f_2}{\partial \dot{\theta}} \end{pmatrix}\Big|_e 
= \begin{pmatrix} 0 & 1 \\ \frac{3g}{2\ell}\sin\theta & -\frac{3b}{m\ell^2} \end{pmatrix}\Big|_e \\
&= \begin{pmatrix} 0 & 1 \\ 0 & -\frac{3b}{m\ell^2} \end{pmatrix} \\
B &= \frac{\partial f}{\partial u}\Big|_e = \begin{pmatrix} \frac{\partial f_1}{\partial \tau} \\ \frac{\partial f_2}{\partial \tau}  \end{pmatrix}\Big|_e 
= \begin{pmatrix} 0 \\ \frac{3}{m\ell^2} \end{pmatrix}\Big|_e \\
&= \begin{pmatrix} 0 \\ \frac{3}{m\ell^2} \end{pmatrix}.
\end{align*} 
Similarly, the output is given by $y=\theta = x_1 \defeq h(x,u)$, which implies that
\begin{align*}
C &= \frac{\partial h}{\partial x}\Big|_e = \begin{pmatrix} \frac{\partial h_1}{\partial \theta} & \frac{\partial h_1}{\partial \dot{\theta}} \end{pmatrix}\Big|_e 
= \begin{pmatrix} 1 & 0 \end{pmatrix}\Big|_e \\
&= \begin{pmatrix} 1 & 0 \end{pmatrix} \\
D &= \frac{\partial h}{\partial u}\Big|_e = 0.
\end{align*}
The linearized state space equations are therefore given by
\begin{align*}
\dot{\tilde{x}} &= \begin{pmatrix} 0 & 1 \\ \frac{3g\sin\theta_e}{2\ell} & -\frac{3b}{m\ell^2} \end{pmatrix} \tilde{x} + \begin{pmatrix} 0 \\ \frac{3}{m\ell^2} \end{pmatrix} \tilde{u} 
\notag \\
\tilde{y} &= \begin{pmatrix} 1 & 0\end{pmatrix} \tilde{x},
\end{align*}
which is similar to Equation~\eqref{eq:dm_arm_linearized_ss} but with linearized state and output.


The transfer function can be found from the linearized state space equations using Equation~\eqref{eq:dm_tf_from_ss} as
\begin{align*}
H(s) &= C(sI-A)^{-1}B+D \\
	&= \begin{pmatrix} 1 & 0 \end{pmatrix} \left(s\begin{pmatrix}1 & 0 \\ 0 & 1 \end{pmatrix} -\begin{pmatrix} 0 & 1 \\ 0 & -\frac{3b}{m\ell^2}\end{pmatrix} \right)^{-1} \begin{pmatrix} 0 \\ \frac{3}{m\ell^2} \end{pmatrix} \\
	&= \begin{pmatrix} 1 & 0 \end{pmatrix} \begin{pmatrix} s & -1 \\ 0 & s+\frac{3b}{m\ell^2}\end{pmatrix}^{-1} \begin{pmatrix} 0 \\ \frac{3}{m\ell^2} \end{pmatrix} \\
	&= \frac{\begin{pmatrix} 1 & 0 \end{pmatrix} \begin{pmatrix} s+\frac{3b}{m\ell^2} & 1 \\ 0 & s \end{pmatrix} \begin{pmatrix} 0 \\ \frac{3}{m\ell^2}\end{pmatrix}}{s^2 + \frac{3b}{m\ell^2}s} \\
	&= \frac{\begin{pmatrix} s+\frac{3b}{m\ell^2} & 1 \end{pmatrix} \begin{pmatrix} 0 \\ \frac{3}{m\ell^2}\end{pmatrix}}{s^2 + \frac{3b}{m\ell^2}s} \\
	&= \frac{\frac{3}{m\ell^2}}{s^2 + \frac{3b}{m\ell^2}s},
\end{align*}
which is identical to Equation~\eqref{eq:soln_a6_2}.
