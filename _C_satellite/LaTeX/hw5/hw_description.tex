For the satellite attitude control problem:
  \begin{description}
    \item[(a)] Start with the linearized equations for the satellite attitude problem and use the Laplace transform to convert the equations of motion to the s-domain.
    \item[(b)] Find the full transfer matrix from the input $\tau(s)$ to the outputs $\Phi(s)$ and $\Theta(s)$.
%    \item[(b)] Find the full transfer matrix from the input $\tau(s)$ to the outputs $\Phi(s)$ and $\Theta(s)$.  Identify the fast and slow subsystem.
	\item[(c)] From the transfer matrix, find the second-order transfer function from $\Theta(s)$ to $\Phi(s)$.
	\item[(d)] Under the assumption that the panel moment of inertia $J_p$ is significantly smaller than the spacecraft moment of inertia $J_s$ (specifically, $(J_s+J_p)/J_s \approx 1$), find the second-order approximation for the transfer function from $\tau(s)$ to $\Theta(s)$.
%    \item[(c)] Find an approximation to the system that is a cascade of a SISO fast system and a SISO slow system, and identify the disturbances that are being ignored.
%    \item[(d)] Argue that the fast-slow cascade approximation makes sense physically.
	\item[(e)] From your results on parts (c) and (d), form the approximate transfer function cascade for the satellite/panel system and justify why it makes sense physically.
  \end{description}